\documentclass{article}
%\usepackage{exsheets}
%\usepackage{enumitem}
\usepackage{fancyhdr}
%\usepackage[mathrm,colour,cntbysection]}
\usepackage{listings}
\usepackage{color}
\usepackage[ answerdelayed]{exercise}
\definecolor{dkgreen}{rgb}{0,0.6,0}
\definecolor{gray}{rgb}{0.5,0.5,0.5}
\definecolor{mauve}{rgb}{0.58,0,0.82}
  

\lstset{frame=tb,
  language=Haskell,
  aboveskip=3mm,
  belowskip=3mm,
  showstringspaces=false,
  columns=flexible,
  basicstyle={\small\ttfamily},
  numbers=none,
  numberstyle=\tiny\color{gray},
  keywordstyle=\color{blue},
  commentstyle=\color{dkgreen},
  stringstyle=\color{mauve},
  breaklines=true,
  breakatwhitespace=true,
  tabsize=3
  }

%% Change this for title information 
\newcommand\ExTitle{First Steps \ }
\newcommand\fullExTitle{Exercises \\ \ExTitle }
\newcommand\footerExTitle{\ExTitle -\  Exercises and Solutions }

\pagestyle{fancy}
\fancyhead{} % clear all header fields
\renewcommand{\headrulewidth}{0pt} % no line in header area
\fancyfoot{} % clear all footer fields
\fancyfoot[LE,RO]{\thepage}           % page number in "outer" position of footer line
\fancyfoot[RE,LO]{\footerExTitle} % other info in "inner" position of footer line

% \usepackage[mathrm,colour,cntbysection]{czt}
%\usepackage{minted}
\begin{document}
\begin{Huge}
	\begin{center}
	\fullExTitle
	\end{center}
\end{Huge}

\begin{Exercise} 
	
  Parenthesise the following numeric expressions:  \\
	\begin{enumerate}
 	 \item
 		2\^{}3*4 \\
	 \item
 		2 * 3 + 4 *5\\
 
	 \item
 		2 + 3 * 4\^{}5\\
	\end{enumerate}  
   Use GHCi to check your answers. 
\end{Exercise}
\begin{Answer}
	\begin{enumerate}
 	 \item
 		Question(Q). 2\^{}3*4 \\
		Answer(A). (2\^{}3) * 4
	 \item
 		Q. 2 * 3 + 4 *5\\
                 A. (2 * 3) + (4 *5)
	 \item
 		Q. 2 + 3 * 4\^{}5\\
		A. 2 + ( 3 * (4 \^{}5))
	\end{enumerate}  
\end{Answer}

\begin{Exercise} 
   The script below contains three syntactic errors. Correct these errors and then check that your script works properly using GHCi. \\ 

\begin{lstlisting}
    n =  let  a = 10
          xs =  [1,2,3,4,5]
          in    a `div` length xs 
\end{lstlisting}
\end{Exercise} 
\begin{Answer}
\begin{lstlisting}
n =  let  a = 10     -- lowercase
          xs =  [1,2,3,4,5]   -- indentation
          in    a `div` length xs -- incorrect quotes
\end{lstlisting}
\end{Answer}
\begin{Exercise} 
The library function ${last}$ selects the last element of a non-empty list; for example:\\
\begin{tabbing}
	\=\hspace{2em}\=\kill
	\>\> $last$ \ [1,2,3,4,5] \ = \ 5 \\
\end{tabbing}
Show that the function $last$ could be defined in terms of the other library functions introduced so far. 
\end{Exercise} 
\begin{Answer}
\begin{lstlisting}
last xs = head (reverse xs)
\end{lstlisting}
\end{Answer}
\pagebreak
\begin{Exercise} 
The library function ${init}$ removes the last element of a non-empty list; for example:\\
\begin{tabbing}
	\=\hspace{2em}\=\kill
	\>\> $init$ \ [1,2,3,4,5] \ = \ [1,2,3,4] \\
\end{tabbing}
Show that the function $init$ could be defined in terms of the other library functions introduced so far. 
\end{Exercise} 
\begin{Answer}
\begin{lstlisting}
myinit xs = reverse(tail (reverse (xs)))
\end{lstlisting}
\end{Answer}
% Uncomment this for the Answers


\end{document}