%%% Uncomment the following for normal slide show
%
\documentclass[ignorenonframetext]{beamer}

%%% or uncomment this for handouts
%\documentclass[handout,ignorenonframetext]{beamer}

%%% or uncomment this for the article version
%\documentclass[11pt]{article}
%\usepackage{beamerarticle}
%
\mode<article>
{
  \usepackage{fullpage}
  \usepackage{pgf}
  \usepackage{hyperref}
  \setjobnamebeamerversion{example.beamer}
}

\mode<presentation>
{
  %\usepackage{listings}
  % \usetheme{Dresden}
  % \usetheme{Marburg}
  \usetheme{Hannover}
 % \usetheme{Singapore}
  % \useoutertheme{smoothbars}
  \setbeamercovered{transparent}
}

\mode<handout>
{
%%% In handout mode give the individual pages a light grey background
\setbeamercolor{background canvas}{bg=black!5}
%%% Put more than one frame on each page to save paper.
\usepackage{pgfpages} 
\pgfpagesuselayout{4 on 1}[letterpaper,border shrink=3mm, landscape]
% \pgfpagesuselayout{2 on 1}[letterpaper,border shrink=5mm, portrait]
% \setbeameroption{show notes}
}
% \usepackage[latin1]{inputenc}

% common pagkages here? 
\usepackage[english]{babel}
\usepackage{listings}


\title{The Caesar Cypher}
\author{Daniel A.\ Graham}
%\subject{Presentation Programs}
%
%\institute[WIT]{
%  Department of Economics\\
%  Duke University}
%

\begin{document}

\frame{\maketitle}

\section{Introduction}

A well known method of encoding a string in order to disguise its contents is the \textit{Caesar Cipher}
, named after its use by Julius Caesar. To encode a string, Caesar simple replaced each each letter in the string by the letter places further down in the alphabet. 
\begin{frame}<presentation> [fragile, label = test]

  \frametitle{Caesar Cipher}

Example of string encoding with constant shift factor of 3 $\dots$
    \pause
 \begin{itemize}
 \item
 "abc " would be encoded to "def"  
 \item
 "haskell is fun" would be encoded to "kdnnhoo lv ixq"

 \end{itemize}
 %
%  \begin{onlyenv}
%  \begin{lstlisting} [language=Haskell]
%  f :: Int -> Char
%  \end{lstlisting}
%  \end{onlyenv}
More Generally 
\article { the specific shift factor of three used by Caesar can be replaced by any integer between one and twenty-five, thereby giving twenty-five different ways of encoding a string. }
So, more generallly with a shift factor of 4, for example:\hfill \break
"abc" would be encoded by "efg" \hfill \break
How will we use Haskell to implement the Caesar and more \ldots
\end{frame}
 

\section{Encoding and decoding}
We will use a  number of standard functions on characters that are provided in a library called $Data.Char$ which can be loaded into a Haskell script by including the following declaration at the start of the script 
 \begin{frame}[fragile, label=encoding]
  \frametitle{Encoding and Decoding} 
   \begin{onlyenv}
  \begin{lstlisting} [language=Haskell]
 import Data.Char   -- imports standard functions on characters
  \end{lstlisting}
  \end{onlyenv}

    \pause
For simplicity, we will only encode the lower-case characters within a string and leave the other characters unchanged. 
Firstly 
\article { $chr$ and $ord$ are Data.Char functions. $chr$ returns a character given its ordinal number. $ord$ returns a given character's ordinal number. }
  \begin{onlyenv}
  \begin{lstlisting} [language=Haskell]
let2Int :: Char -> Int
let2Int c = ord c - ord 'a' 

int2Let :: Int -> Char
int2Let n = chr (ord 'a' + n)
 \end{lstlisting}
  \end{onlyenv}

\end{frame}

This section has a second frame as well.

\frame[label=secondframeb]{
  \frametitle{Second section, second frame with two overlays.}

  \begin{itemize}
  \item The first item$\dots$
    \pause
  \item $\dots$ and the second one.
  \end{itemize}
}

\section{The third section}

This is the third section of the article version. In the
presentation, there is a frame containing two overlays. 

\frame[label=thirdframe]{
  \frametitle{Third section, first and only frame with two overlays.}

  \begin{itemize}
  \item The first item$\dots$
    \pause
  \item $\dots$ and the second one.
  \end{itemize}
}

\section{The fourth section}

This is the fourth section of the article version. In the
presentation, there is a frame containing an overlay. 

\frame[label=fourthframe]{
  \frametitle{Fourth section, first and only frame with two overlays.}

  \begin{itemize}
  \item The first item$\dots$
    \pause
  \item $\dots$ and the second one.
  \end{itemize}
}

\section{The fifth section}

This is the fifth section of the article version. In the
presentation, there is a frame containing an overlay. 

\frame[label=fifthframe]{
  \frametitle{Fifth section, first and only frame with three overlays.}

  \begin{enumerate}
  \item There are five sections altogether.
    \pause
  \item With a total of 7 frames (pages in handout mode).
	\pause
  \item And 14 overlays (pages in presentation mode).
  \end{enumerate}
}

\end{document}
